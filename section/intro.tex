%%%%%%%%%%%%%%%%%%%%%%%%%
\section{Introduction}
%%%%%%%%%%%%%%%%%%%%%%%%%

Social influence is an important mechanism that regulates the dynamics of social networks. Social influencers are users who regularly produce authoritative or novel content on specific topics and who can reach a wide number of followers. Finding social influencers has become a fundamental task in many online applications, ranging from brand marketing
~\cite{bond201261,richardson2002mining,van2007new} to opinion mining \cite{pang2008opinion,li2012mining}, expert finding for question 
answering~\cite{riahi2012finding}, misinformation propagation~\cite{DBLP:conf/icde/SongHL17} or presidential elections~\cite{DBLP:journals/nature/BondFJKMSF12}.
%task execution \pcm{rephrase this one, unclear what this means}~\cite{sun2014analyzing,miao2010generative}. 

The task of finding social influencers is, however, extremely challenging due to the subjectivity in perceiving social influence and the requirement for expert knowledge in determining the authenticity of user-generated content. Existing work mainly tackles the problem using supervised learning approaches that rely on a training set hand-labeled by domain experts \cite{Cheng2014,Lehmann2013,wei2016learning}. Such an approach can be viewed as a way of transferring expert knowledge to learning models through the training data. While models trained in this fashion are effective at finding social influencers who are similar to those in the training data, they are intrinsically limited by the availability of expert labels, which are typically very hard to gather and hence highly limited. As an example, our collaboration with a major fashion retailer reveals that an expert can only recognize no more than 200 fashion influencers on Twitter over a 3-week period of time. Finding social influencers is, therefore, a long and usually laborious process even for domain experts.

Compared to an individual expert, online crowds 
%-- as the \emph{direct influencee} in a social network -- 
possess as a whole a broader knowledge of social influencers in various domains, e.g., fashion, fitness, or information technology. Therefore, we advocate in this paper a human computation approach that crowdsources the task of finding social influencers in the form of open-ended question-answering. Specifically, we consider a task where the crowd is asked to name as many as possible social influencers in a predefined domain. By aggregating the answers from a large number of crowd workers, we can identify the identities (e.g., usernames on Twitter) of a large number of social influencers in a efficient and cost-effective manner. 

Despite its obvious benefit, aggregating open-ended answers from the crowd is however challenging: individual crowd workers may only possess fragmented knowledge that is of low-quality. While answer aggregation has been extensively studied in human computation \cite{dawid1979maximum,whitehill2009whose,ZhengLLSC17}, we note that existing methods are not applicable in our context of open-ended answers. Previous methods mainly consider classification problems, where crowd workers are asked to classify an existing \emph{closed} pool of data instances into \emph{pre-defined classes}. While in our context, we consider an \emph{open} pool of answers that are \emph{all deemed relevant} by crowd workers. How to aggregate open-ended answers provided by crowd workers with various knowledge levels remains, therefore, a key open research problem.

To address this problem, we introduce \sys, a new probabilistic learning and inference framework based on Expectation Maximization (EM) for open-ended answers aggregation. Our framework models both the reliability of workers and the quality of the answers as latent variables, and infers these variables in an iterative and mutually boosting manner through EM. A small number of labeled answers is used as the seed for starting the inference -- our framework can thus be viewed as an extension of ``learning-from-crowds''  \cite{raykar2010learning,tian2012learning,yang2018leveraging} to the semi-supervised learning scenario. Given the different number of answers contributed by workers, we propose a Bayesian version of the framework and a variational inference algorithm that characterize the uncertainty in inferring worker reliability, which improves both the robustness of the framework as well as its interpretability. 

In summary, we make the following key contributions:
\begin{itemize}
\item We formally define the problem of finding social influencers through open-ended answers aggregation;
\item We introduce a new EM-based learning and inference framework for open-ended answers aggregation; and
\item We present a Bayesian version of the framework together with an efficient variational inference algorithm for parameter estimation. 
\end{itemize}

To the best of our knowledge, we are the first to adopt an open-ended answers aggregation approach for finding social influencers. Our extensive empirical evaluation on two domains -- fashion and information technology -- demonstrates that \sys substantially improves the state of the art by 40.48\% AUC.

