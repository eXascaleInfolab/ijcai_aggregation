%%%%%%%%%%%%%%%%%%%%%%%%%
\section{Introduction}
%%%%%%%%%%%%%%%%%%%%%%%%%

Social influence is an important mechanism that regulates the dynamics of social networks. Social influencers are those users who regularly produce authentic content on a specific topic and can reach a large number of followers. Finding social influencers has become a fundamental task in many online applications, ranging from brand marketing
~\cite{bond201261,richardson2002mining,van2007new} to opinion mining \cite{pang2008opinion,li2012mining} and expert finding for question 
answering~\cite{riahi2012finding}, misinformation propagation~\cite{DBLP:conf/icde/SongHL17} and task execution~\cite{sun2014analyzing,miao2010generative}. \jie{Can we have more recent reference?} 

The task of finding social influencers is, however, challenging due to the subjectivity in perceiving social influence and the requirement for specialist knowledge in determining the authenticity of user-generated content. Existing work mainly tackles the problem using the supervised learning approach that relies on a training set hand-labeled by domain experts \cite{Cheng2014,Lehmann2013,wei2016learning}. \jie{Also need more recent referenc.} Such an approach can be viewed as a way of transferring expert knowledge to learning models through the training data. While training models in this fashion is scalable in finding similar social influencers as those in the training data, they are intrinsically limited by the availability of expert labels, which is often highly limited. As an example, our collaboration with a major European fashion company reveals that an expert can only recognize no more than 200 fashion influencers on Twitter over a 3-month period of time. Finding social influencers is, therefore, a long and usually laborious process even for domain experts.

Compared to an individual expert, the crowd -- as the \emph{direct influencee} in a social network -- possess a more broad knowledge of social influencers in various domains, e.g., fashion, fitness, and information technology. We, therefore, advocate a human computation approach that crowdsources the task of finding social influencers in the form of open-ended question-answering. Specifically, we consider a task where the crowd is asked to name, as many as possible, the social influencers in a predefined domain. By aggregating the answers from a large number of crowd workers, we can collect the usernames of a big number of social influencers in a efficient and cost-effective manner. 

Despite the obvious benefit, aggregating open-ended answers from the crowd is however challenging: individual crowd workers may only possess fragmented knowledge that is of low-quality. While answer aggregation has been extensively studied in human computation \cite{dawid1979maximum,whitehill2009whose,ZhengLLSC17}, we note that existing methods are not applicable in our context of open-ended answers. These methods mainly consider the classification problem, where crowd workers are asked to classify a existing, \emph{closed} pool of data instances into \emph{pre-defined classes}. While in our context, we consider an \emph{open} pool of answers that are \emph{all deemed as relevant} by crowd workers. It therefore, remains a key open research question how to aggregate open-ended answers contributed by crowd workers with various knowledge levels.

To address this problem, we introduce \sys, a new probabilistic learning and inference framework based on Expectation Maximization (EM) for open-ended answer aggregation. Our framework models both worker reliability and answer quality as latent variables, and   infers these variables in an iterative and mutually boosting manner through EM. A small number of labeled answers is used as the seed for starting the inference -- our framework can thus be viewed as an extension of ``learning from crowds''  \cite{raykar2010learning,yan2010modeling,tian2012learning} to the semi-supervised learning scenario. Given the different number of answers contributed by workers, we propose a Bayesian version of the framework and a variational inference algorithm that characterizes the uncertainty in inferring worker reliability, which improves both the robustness of the framework as well as the interpretability. 

In summary, we make the following key contributions:
\begin{itemize}
\item We formally define the problem of finding social influencers through open-ended answer aggregation;
\item We introduce a new EM-based learning and inference framework for open-ended answer aggregation; and
\item We present a Bayesian version of the framework together with an efficient variational inference algorithm for parameter estimation. 
\end{itemize}

To the best of our knowledge, our work is the first to study an open-ended answer aggregation approach for finding social influencers. Our extensive experiments on two domains -- fashion and information technology -- demonstrate that \sys substantially improves the state of the art by XXX\% accuracy and XXX\% AUC. 