%%%%%%%%%%%%%%%%%%%%%%%%%
\section{Related Work}
%%%%%%%%%%%%%%%%%%%%%%%%%

In this section, we briefly discusses related work on answers aggregation and social influence analysis.

\smallskip
\noindent\textbf{Answers Aggregation.} Aggregating workers' answers is a central problem in crowdsourcing. Typical methods include majority voting \cite{sheng2008get} and those based on EM, which simultaneously estimate the true labels and parameters related to the annotation process. A classical work was carried out by \citeauthor{dawid1979maximum} (\citeyear{dawid1979maximum}), who propose to model worker reliability as a confusion matrix. More recently, \citeauthor{demartini2012zencrowd} (\citeyear{demartini2012zencrowd}) propose a similar method ZenCrowd, which also applies EM to estimate the worker reliability -- however as a single parameter for each worker. \citeauthor{whitehill2009whose} (\citeyear{whitehill2009whose}) propose the GLAD model that goes beyond worker reliability to further consider task difficulty. Closer to our method is LFC proposed by \citeauthor{raykar2010learning} (\citeyear{raykar2010learning}), who consider to model worker reliability as a latent variable with a prior distribution. Unlike LFC, our proposed framework further incorporates existing labels and social features, thus extending the applicability of answer aggregation to open-ended question-answering tasks. We note that our method is also related to the ``learning-from-crowds'' line of research \cite{raykar2010learning,tian2012learning,yang2018leveraging}, which considers the classification problem with noisy labels contributed by the crowd. Our framework is different in that it only requires a subset of the data instances to be labeled; it can, therefore, be characterized as a semi-supervised learning-from-crowd approach. 

% Li et al.~\cite{catd} proposed a weighted majority voting. In their technique, 
% worker's quality is assessed by two measures: 
% a probability that reflects her reliability and a confidence based on the number of tasks she
% answered. This technique aims at assigning a higher quality for workers who give 
% plenty of answers close to the truth.
% In~\cite{dawid1979maximum}, the authors proposed a probabilistic model 
% where a confusion matrix is used to model the worker's quality and it 
% applies the EM framework to infer the truth.  a probability (a real number between 0 and 1).
% An extension to ZenCrowd was proposed in~\cite{li2014resolving} where
% worker's quality is modeled in a wider range (in $(-\infty,+\infty)$.
% These methods aims at maximizing the likelihood that reliable workers tend to 
% provide correct results.\iar{not confident about this last sentence}
% More recent work~\cite{yin2017aggregating} uses variational autoencoders
% to infer the true label  of objects in an unsupervised manner.
% \iar{I am adding this as it is an IJCAI'17 but it is not clear to me what is the main 
% contribution other then using autoencoders for aggregation}

\smallskip
\noindent\textbf{Social Influence Analysis.} \jie{this part needs to be reformulated.}
%To enable our method to be able to identify real influencers, we incorporate in our model
%a neural network to learn influencer's properties. 
Several approaches have been proposed 
to measure the influence of users in a social media platforms.
\cite{Cheng2014} proposes a geo-spatial-driven approach to find experts 
by using the GPS coordinates of millions of Twitter users. In this approach, user's influence
is a combination of her local authority and her topical authority reflecting respectively
her influence on the audience close to her in distance and in a specific topic.
\cite{Lehmann2013} define a framework to classify users as influencers 
based on three main features: (1) their visibility measured by the number of followers and tweeter
lists containing them, (2) their tweeting activity measured by the number of tweets and retweets 
per day and (3) their topic focus measured by the number of articles tweeted 
about a specific topic. 
\cite{wei2016learning} develop a probabilistic method to find topic experts in Twitter. They
measure the influence of users based on her local relevance and three main types of relation: 
follower relation, user-list relation and list-list relation. 
They assign to each user a score based on the computed influence and
define the influencers as the top-N users with the highest scores.
\iar{help me to add limitations of these methods plus why we are better}  