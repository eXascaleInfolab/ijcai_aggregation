%%%%%%%%%%%%%%%%%%%%%%%%%
\section{Related Work}
%%%%%%%%%%%%%%%%%%%%%%%%%

In this section, we first briefly discuss related work on finding social influencers and then review existing answers aggregation methods. Existing methods for finding social influencers are mainly feature-based. Typical features that have been explored include meta-data features such as the number of followers and followees~\cite{Lehmann2013,Cheng2014}, semantic features such as the topics of a candidate influencer's microposts \cite{riahi2012finding,wei2016learning}, or the activeness of a candidate influencer in online activities \cite{agarwal2008identifying,Lehmann2013}. These methods, however, all rely on expert labels which are difficult to obtain. Compared to these feature-based techniques, our work tackles the problem using an orthogonal dimension, i.e., through crowdsourced open-ended question-answering. Being an aggregation of crowd answers technique, our proposed framework is able to leverage work reliability and social features at a time.  


Answers aggregation is a central problem in crowdsourcing. Typical methods include majority voting \cite{sheng2008get} and those based on EM, which simultaneously estimate the true labels and parameters related to the annotation process. We have introduced and compared typical EM-based methods in Section~\ref{sec:result}, including the classic DS model proposed by \citeauthor{dawid1979maximum} (\citeyear{dawid1979maximum}), and the more recent ZenCrowd by \citeauthor{demartini2012zencrowd} (\citeyear{demartini2012zencrowd}) and GLAD by \citeauthor{whitehill2009whose} (\citeyear{whitehill2009whose}). More similar to our method is LFC proposed by \citeauthor{raykar2010learning} (\citeyear{raykar2010learning}), which models worker reliability as a latent variable with a prior distribution. Unlike LFC, our proposed framework further incorporates existing labels and social features, thus extending the applicability of answer aggregation to open-ended question-answering tasks. We note that our method is also related to the ``learning-from-crowds'' line of research \cite{raykar2010learning,tian2012learning,yang2018leveraging}, which considers the classification problem with noisy labels contributed by the crowd. Our framework is different in that it only requires a subset of the data instances to be labeled; it can, therefore, be characterized as a semi-supervised learning-from-crowd approach. 