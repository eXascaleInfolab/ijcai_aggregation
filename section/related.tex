%%%%%%%%%%%%%%%%%%%%%%%%%
\section{Related Work}
%%%%%%%%%%%%%%%%%%%%%%%%%

In this section, we briefly discusses related work on answers aggregation and social influence analysis.

\smallskip
\noindent\textbf{Answers Aggregation.} Aggregating workers' answers is a central problem in crowdsourcing. Typical methods include majority voting \cite{sheng2008get} and 

Li et al.~\cite{catd} proposed a weighted majority voting. In their technique, 
worker's quality is assessed by two measures: 
a probability that reflects her reliability and a confidence based on the number of tasks she
answered. This technique aims at assigning a higher quality for workers who give 
plenty of answers close to the truth.
In~\cite{dawid1979maximum}, the authors proposed a probabilistic model 
where a confusion matrix is used to model the worker's quality and it 
applies the EM framework to infer the truth. Similarly, ZenCrowd 
\cite{demartini2012zencrowd} applies an EM framework to compute the 
worker's quality modeled as a probability (a real number between 0 and 1).
An extension to ZenCrowd was proposed in~\cite{li2014resolving} where
worker's quality is modeled in a wider range (in $(-\infty,+\infty)$.
These methods aims at maximizing the likelihood that reliable workers tend to 
provide correct results.\iar{not confident about this last sentence}
More recent work~\cite{yin2017aggregating} uses variational autoencoders
to infer the true label  of objects in an unsupervised manner.
\iar{I am adding this as it is an IJCAI'17 but it is not clear to me what is the main 
contribution other then using autoencoders for aggregation}

\subsection{Social Influence Analysis}
%To enable our method to be able to identify real influencers, we incorporate in our model
%a neural network to learn influencer's properties. 
Several approaches have been proposed 
to measure the influence of users in a social media platforms.
\cite{Cheng2014} proposes a geo-spatial-driven approach to find experts 
by using the GPS coordinates of millions of Twitter users. In this approach, user's influence
is a combination of her local authority and her topical authority reflecting respectively
her influence on the audience close to her in distance and in a specific topic.
\cite{Lehmann2013} define a framework to classify users as influencers 
based on three main features: (1) their visibility measured by the number of followers and tweeter
lists containing them, (2) their tweeting activity measured by the number of tweets and retweets 
per day and (3) their topic focus measured by the number of articles tweeted 
about a specific topic. 
\cite{wei2016learning} develop a probabilistic method to find topic experts in Twitter. They
measure the influence of users based on her local relevance and three main types of relation: 
follower relation, user-list relation and list-list relation. 
They assign to each user a score based on the computed influence and
define the influencers as the top-N users with the highest scores.
\iar{help me to add limitations of these methods plus why we are better}  