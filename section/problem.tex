%%%%%%%%%%%%%%%%%%%%%%%%%
\section{Problem Formulation}
%%%%%%%%%%%%%%%%%%%%%%%%%

In this section, we first introduce the notations used in the paper and then formally define the open-ended answer aggregation problem.

\smallskip
\noindent\textbf{Notations.} Throughout this paper, we use boldface lowercase letters to denote vectors and boldface uppercase letters to denote matrices. For an arbitrary matrix $\mathbf{M}$, we use $\mathbf{M}_{i,j}$ to denote the entry at the $i$-th row and $j$-th column. We use capital letters (e.g., $\mathcal{P}$) in calligraphic math font to denote sets.

We use $\mathcal{I} = \{i_1, \ldots, i_n\}$ to denote the set of items, and  $\mathcal{J} = \{j_1, \ldots, j_m \}$ to denote the set of workers. For each item $i \in \mathcal{I}$, we have its features organized as a vector $\mathbf{x}_i$; and similarly, each worker $j \in \mathcal{J}$ has a feature vector $\mathbf{y}_j$. We use $\mathbf{A}_{i,j}=1$ to denote that item $i$ is an answer provided by worker $j$, and $\mathbf{A}_{i,j}=0$ otherwise. $\mathbf{A}_{i,j}$ is a sparse matrix where only a small proportion of the entries is non-zero. Our goal is find those items in $\mathcal{I}$ that are of high-quality, denoted by $\mathcal{SI} \subset \mathcal{I}$. We use $z_i = 1$ to denote $i\in \mathcal{SI}$, and $z_i = 0$ otherwise.

\smallskip
\noindent\textbf{Problem Statement.} Given $\mathbf{x}_i$ for each $i\in \mathcal{I}$ and $\mathbf{y}_j$ for each $j\in \mathcal{J}$, and the answer matrix $\mathbf{A}$, our goal is to infer $z_i$ for all $i\in \mathcal{I}$.