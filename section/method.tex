%%%%%%%%%%%%%%%%%%%%%%%%%
\section{The \sys Framework}
%%%%%%%%%%%%%%%%%%%%%%%%%

\sys is a probabilistic framework that aggregates open-ended answers while taking into account worker reliability. In this section, we first introduce how \sys models the open-ended question-answering process, and then introduce the Expectation Maximization method for learning the parameters. Subsequently, we present a Bayesian version of our framework with an efficient variational inference algorithm for characterizing the uncertainty of worker relability. 

\subsection{\sys as a Generative Model}
To infer the true label of a candidate social influencer, we adopt a Bayesian network to describe the open-ended question-answering process as a generative process. In this process, the true label of a candidate influencer follows a Bernoulli distribution:
%
\begin{equation}
    z_i \sim Ber(\theta_i), \theta_i  = \sigma (f^{\mathcal{W}_I}(\mathbf{x}_i)),
    \label{eq:dis_item}
\end{equation}
%
where $\theta_i$ is the parameter of the distribution predicted by the social features of the candidate influencer through a linear model or a neural network, denoted by $f(\cdot)$; $\mathcal{W}_I$ is the parameter of the model; $\sigma$ is the sigmoid function. Note that $\mathcal{W}_I$ is shared among all candidate influencers, thus amortizing the inference \cite{gershman2014amortized} by exploiting the similarity among candidate influencers. 


We then model worker reliability by $r_j \in [0,1]$ $(j\in \mathcal{J})$, where $r_j=1$ indicates the worker is fully reliable and $r_j=0$ indicates the worker is not reliable. We use the reliability of a worker to define the probability of her named candidate influencer being a true influencer:
%
\begin{equation}
    p(\mathbf{A}_{i,j} | z_i,  r_j) = r_j^{\mathbbm{1}[z_i=\mathbf{A}_{i,j}]} (1- r_j)^{\mathbbm{1}[z_i\ne \mathbf{A}_{i,j}]},
    \label{eq:item_worker}
\end{equation}
%
where $\mathbbm{1}(\cdot)$ is an indicator function \jie{check on papers}. Note that Equation~\ref{eq:item_worker} considers a worker to be reliable if she does not name a candidate influencer who turns out not be a not real influencer. It is, however, likely that a worker does not name a candidate influencer $i$ simply because she did not think of $i$. That means that we can only partly treat the non-named candidate influencers as those the worker considers as non-influencers. It is, therefore, necessary to introduce negative sampling into the inference algorithm, which we introduce in Section~\ref{sec:em}. 

Worker reliability $r_j$ is so far assumed to be a fixed parameter. In practice, we would like to distinguish our \emph{confidence} in estimating the reliability of the workers providing different numbers of answers: we should be more confident in estimating the reliability of workers who provide 50 answers than those who only provide 5 answers. To quantify the confidence in our inference, we adopt a Bayesian treatment of $r_j$ by introducing a prior:
%
\begin{equation}
        r_j \sim Beta(A,B).
        \label{eq:rj_dist}
\end{equation}
%
where $A$ and $B$ are the parameters of the prior distribution. The incorporation of confidence will make our framework more robust to overfitting, as we show in our experiments. 

The overall \sys framework is depicted in Figure~\ref{fig:graphical_model}. Model learning concerns parameter learning for $\mathcal{W}_I$ (and $r_j$ when modeled as a fixed parameter) and posterior inference for $z_i$ (and $r_j$ when modeled as a latent variable).  

\begin{figure}[htb] 
{
\begin{tikzpicture}

  % Define nodes
  \node[obs,double]          (a)   {$\mathbf{A}_{i,j}$}; %
  \node[latent, above left=1.2 of a]  (z)   {$z_i$}; %
  \node(rect) [latent, dashed, above right=1.2 of a]  (r)   {$r_j$};
  \node (rect)  [draw,dashed,minimum width=0.5cm,minimum height=0.5cm, above right=1.2 of r] (alpha){$A$};
  \node (rect)  [draw,dashed,minimum width=0.5cm,minimum height=0.5cm, right=1.2 of r] (beta){$B$};


  \node(rect)  [draw,thick,minimum width=0.5cm,minimum height=0.5cm, left=2 of z] (w) {$\mathcal{W}_I$} ; 
  \node[obs,double, above left=1.2 of z]  (x) {$\mathbf{x}_i$} ; %


  % Factors
  \edge {alpha,beta} {r} ;
  \edge {w, x} {z} ;
  \edge {z,r} {a} ;

  \plate {yx} {(a)(r)} {Worker} ;
  \plate {} {(a)(z) (x) (yx.north west)(yx.south west)} {Candidate Influencer} ;

\end{tikzpicture}
}
\caption{Graphical model representation of \sys. Double (greyed) circles represent observed variables, while single circles represent latent variables.  Squares represent model parameters. We note that worker reliability $r_j$ can either be a fixed parameter or a latent variable (thus, it is denoted by a dashed circle in the figure). Edges represent conditional relationship in answer generation.}
\label{fig:graphical_model}
\end{figure}


\subsection{Expectation Maximization for \sys} 
\label{sec:em}
For model learning, we first consider the case when worker reliability is modeled as a fixed parameter. Due to the latent variable $z_i$, we use Expectation Maximization (EM) \cite{dempster1977maximum} for parameter estimation. Note that the EM method introduced below also serves as the basis for our variational inference method for the case when worker reliability is modeled as a latent variable. The EM method iteratively takes 1) the E-step to infer the posterior distribution over the true label $z_i$ ($i\in \mathcal{I}$) given the current estimates of model parameters, and 2) the M-step to learn the parameter $\mathbf{W}_I$ and $r_j$ ($j\in \mathcal{J}$) given the newly inferred latent variable.

We first introduce negative sampling, which affects parameter estimation in EM. For each worker $j\in \mathcal{J}$, we consider all candidate influencers named by her, and a random sample of candidate influencers she does not name -- obtained by negative sampling, as relevant candidate influencers in estimating her reliability. We denote such worker-relevant candidate influencers as $\mathcal{I}_j$. Similarity, we use $\mathcal{J}_i$ to denote the relevant workers in estimating the quality of the candidate influencer $i$ when negative sampling is applied. \jie{check if it's clear}

\smallskip
\noindent\textbf{E-step.} We first denote $\mathbf{A}_{i,\mathcal{J}_i}$ as all relevant answers to candidate influencer $i$, i.e., $\mathbf{A}_{i,\mathcal{J}_i} = \{\mathbf{A}_{i,j}|j \in \mathcal{J}_i\}$. The E-step then infers the true label of a candidate influencer $i$ as follows:
%
\begin{align}
    p(z_i) & \delequal p(z_i|\mathbf{A}_{i,\mathcal{J}_i},r_{j},x_i;\mathcal{W}_I) \nonumber\\
        &\propto p(\mathbf{A}_{i,\mathcal{J}_i}|z_i,r_j) p(z_i|x_i;\mathcal{W}_I) \label{eq:dist_cond}\\
        &=p(z_i|x_i;\mathcal{W}_I) \prod_{j \in \mathcal{J}_i} p(\mathbf{A}_{i,j}|z_i,r_{j}). \nonumber    
\end{align}
%
The inference of $z_i$ is therefore determined by both the reliability of the relevant workers' answers and the social features of the candidate influencer $i$. Note that $p(z_i|x_i,\mathcal{W}_I)$ and $p(\mathbf{A}_{i,j}|z_i,r_{j})$ are defined by Equation~\ref{eq:dis_item} and \ref{eq:item_worker}, respectively.

\smallskip
\noindent\textbf{M-step.}
Given the true labels of candidate influencers inferred by the E-step, the M-step maximizes the following log-likelihood function to learn the parameters:
\begin{align}
   \mathcal{L} 
    &=\sum_{z_i}p(z_i)\log p(\mathbf{A}_{i,j},z_i|r_j,\mathbf{x}_i;\mathcal{W}_I)\nonumber\\
    &=\sum_{z_i}p(z_i)\log [p(\mathbf{A}_{i,j}| z_i , r_j)  p(z_i |\mathbf{x}_i;\mathcal{W}_I)] \label{eq:likelihood_m}\\
    &=\underbrace{\sum_{z_i}p(z_i) \log p(\mathbf{A}_{i,j}| z_i , r_j)}_{\mathcal{M}_1}
    +\underbrace{\sum_{z_i}p(z_i)\log p(z_i |\mathbf{x}_i;\mathcal{W}_I)}_{\mathcal{M}_2},
    \nonumber
\end{align}
where $p(z_i)$ is obtained by Equation~\ref{eq:dist_cond}. We observe that the M-step can be decomposed into two parts independent of each other. The first part, $\mathcal{M}_1$, can be solved via stochastic gradient ascent (SGA). The gradient of $r_j$ is given by:
%
\begin{equation}
    \frac{\partial \mathcal{M}_1}{\partial r_j}=\frac{p(\mathbf{A}_{i,j}=z_i)}{r_j}-\frac{p(\mathbf{A}_{i,j}\neq z_i)}{1-r_j}.
    \label{eq:grad_rj}
\end{equation}
%
The second part, $\mathcal{M}_2$, interestingly, is exactly the inverse of the cross-entropy between $p(z_i)$ and $p(z_i |\mathbf{x}_i;\mathcal{W}_I)$, which is widely used as the loss function for many classifiers. $\mathcal{M}_2$ can, therefore, be optimized using standard methods (e.g., back-propagation if a neural network is used by \sys).

\subsection{Variational Inference for \sys}
\label{sec:vi}

We now introduce our variational inference method for the Bayesian version of \sys, i.e., when worker reliability is modeled as a latent variable with a prior distribution. In this case, a closed form solution for the EM updates is no longer possible. We develop a meanfield variational approach to approximate the posterior distribution over all hidden variables given the data, which in our model is $p(\mathbf{z},\mathbf{r} | \mathcal{D})$, where $\mathbf{z}$ and $\mathbf{r}$ are the latent true labels for all candidate influencers and for the reliability of all workers, respectively, and $\mathcal{D}$ is the observed data, i.e., $\mathcal{D}=\{\mathbf{A}, \mathbf{x}_i\ (\forall i\in\mathcal{I}), y_i\ (\forall i\in \mathcal{LI})\}$. 

The main idea of variational inference \cite{tzikas2008variational} is to approximate $p(\mathbf{z},\mathbf{r} | \mathcal{D})$ by a variational distribution $q(\mathbf{z},\mathbf{r})$ and perform optimization to minimize the KL divergence between the two distributions. Meanfield variational inference allows for efficient optimization by assuming that $q(\mathbf{z},\mathbf{r})$ factorizes over the latent variables:
\begin{equation}
    q(\mathbf{z},\mathbf{r})=\prod_{i} q(z_i) \prod_j q(r_j).
    \label{eq:dist_fact}
\end{equation}
%
We further assume the forms of the factor functions:
\begin{align}
    q(z_i)=Ber(\phi_i), q(r_j)=Beta(\alpha_j,\beta_j),
\end{align}
%
where $\phi_i$, $\alpha_j$ and $\beta_j$ are variational parameters used to perform optimization to minimize the KL-divergence. This optimization problem can be solved via coordinate ascent \cite{blei2017variational}: updating one factor while keeping all others fixed and iterating until convergence. In the following, we derive the update equations for the latent variables $z_i$ and $r_j$.

\smallskip
\noindent\textbf{Updating $z_i$.} For $z_i$, we have:
%
\begin{equation}
q(z_i)    \propto \prod_{j \in \mathcal{J}_{i}} \exp{\{\mathbb{E}_{q(r_j)}[\log (P(\mathbf{A}_{i,j}|z_i,r_j))]\}}   \times P(z_i|x_i, \mathcal{W}_I). \\
\end{equation}
%
We distinguish between the following two cases:
%
\begin{align}
  1)~ q(z_i=0)   &= \prod_{j \in \mathcal{J}_{i}} \exp{\{\mathbb{E}_{q(r_j)}[\log (1-r_j)]\}}\times (1-\theta_i) \nonumber, \\
  2)~ q(z_i=1)   &= \prod_{j \in \mathcal{J}_{i}} \exp{\{\mathbb{E}_{q(r_j)}[\log (r_j)]\}}\times \theta_i .
\label{eq:q_two_poss}                
\end{align}
%
The expectations can be evaluated as follows:
\begin{align}
    \mathbb{E}_{q(r_j)}[\log (1-r_j)]&= \Psi(\beta_j)-\Psi(\alpha_j+\beta_j), \nonumber \\
    \mathbb{E}_{q(r_j)}[\log (r_j)]&= \Psi(\alpha_j)-\Psi(\alpha_j+\beta_j),
    \label{eq:expect}
\end{align}
where $\Psi(\cdot)$ is the Digamma function. Therefore, the update equation for $z_i$ can be simplified as:
\begin{align}
    q(z_i=0)   &\propto \prod_{j \in \mathcal{J}_{i}} \exp{\{\Psi(\beta_j)-\Psi(\alpha_j+\beta_j)\}}\times (1-\theta_i), \nonumber \\  
     q(z_i=1)    &\propto \prod_{j \in \mathcal{J}_{i}} \exp{\{ \Psi(\alpha_j)-\Psi(\alpha_j+\beta_j)\}}\times \theta_i  .
     \label{eq:qzi}
\end{align}

\smallskip
\noindent\textbf{Updating $r_j$.}  For $r_j$, we have:
%
\begin{equation}
    q(r_j) \propto \prod_{i \in \mathcal{I}_j} \exp{\{\mathbb{E}_{q(z_i)}[\log{P(\mathbf{A}_{i,j}|r_j,z_i)}]\}} \times   {P(r_j)},
\end{equation}
%
where $P(r_j)$ is the variational $r_j$ distribution from last iteration. The expectation can be evaluated as follows:
%
\begin{align}  
& \exp{\{\mathbb{E}_{q(z_i)}[\log{P(\mathbf{A}_{i,j}|r_j,z_i)}]\}} = \nonumber \\
& 
\left\{  
    \begin{array}{lr}
        r_j^{\theta_i}  \times (1-r_j)^{(1 - \theta_i)}, &  \text{if\ } \mathbf{A}_{i,j}=1,\\  
        r_j^{(1 - \theta_i)}  \times (1-r_j)^{\theta_i}, &  \text{if\ } \mathbf{A}_{i,j}=0.   
    \end{array}  
\right. 
\end{align}
%
Therefore, $r_j$ can be efficiently inferred as follow:
%
\begin{align}  
& q(r_j)  \propto \nonumber \\
& 
\left\{  
    \begin{array}{lr}
        Beta(\alpha_j+\sum_{i\in \mathcal{I}_j}  \theta_i,\beta_j+ \sum_{i\in \mathcal{I}_j} (1 - \theta_i) ), &  \text{if\ } \mathbf{A}_{i,j}=1,\\  
        Beta(\alpha_j+\sum_{i\in \mathcal{I}_j} (1 - \theta_i),\beta_j+ \sum_{i\in \mathcal{I}_j} \theta_i), &  \text{if\ } \mathbf{A}_{i,j}=0.   
    \end{array}  
\right. 
\label{eq:qrj}
\end{align}



The update rules described above play the role of the E-step in a variational EM approach. 
The M-step for learning $\mathbf{W}_I$ can be performed in a similar manner as in the previous section. The overall coordinate ascent algorithm is presented in Algorithm~\ref{alg:vi-algo}. It iterates over the E-step (row 2-5) and M-step (row 6-7) until the evidence lower bound (ELBO) \cite{blei2017variational} -- equivalent to the KL-divergence as the optimization objective yet easily computable -- converges. We note that this algorithm has a similar computational complexity as the normal EM method: both Equations~\ref{eq:qzi} and \ref{eq:qrj} can be efficiently computed; and $\mathbf{W}_I$ can be efficiently updated in an incremental manner, i.e., it can learn for a few more epochs starting from the parameter learned in the previous iteration.

\begin{algorithm2e}
    \SetKwInOut{Input}{Input}
    \SetKwInOut{Output}{Output}
    \SetKwInOut{Initialize}{Initialize}
    \Input{$\mathbf{A}, \mathbf{x}_i\ (\forall i\in\mathcal{I}), y_i\ (\forall i\in \mathcal{LI})$}
    \Output{Variational distributions: $q(z_i)$ and $q(r_j)$}
    \Initialize {Variational parameters: $\theta_i$, $\alpha_j=A$, $\beta_j=B$; parameter of the influencer predictor ($f$): $\mathbf{W}_I$}
    \While{the ELBO has not converged}{
    % E-step:\\
        \For {$i \in \mathcal{I}$}
        {
                update $q(z_i)$ using Equation~\ref{eq:qzi}\;
        } 
        \For {$j \in \mathcal{J}$}
        {
                update $q(r_j)$ using Equation~\ref{eq:qrj}\;
        }
    % M-step: \\
        \For {$i \in \mathcal{I}$}
        {
         Update $\mathcal{W}_I$ via standard gradient descent\;
       } 
    }
    \caption{Coordinate Ascent Variational Inference}
    \label{alg:vi-algo}
\end{algorithm2e}