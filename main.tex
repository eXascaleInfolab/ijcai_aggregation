%%%% ijcai19.tex

\typeout{Aggregation for open-ended task}

% These are the instructions for authors for IJCAI-19.

\documentclass{article}
\pdfpagewidth=8.5in
\pdfpageheight=11in
% The file ijcai19.sty is NOT the same than previous years'
\usepackage{ijcai19}

% Use the postscript times font!
\usepackage{times}
\usepackage{soul}
\usepackage{url}
\usepackage[hidelinks]{hyperref}
\usepackage[utf8]{inputenc}
\usepackage[small]{caption}
\usepackage{graphicx}
\usepackage{amsmath}
\usepackage{booktabs}
\usepackage{algorithm}
\usepackage{algorithmic}
\urlstyle{same}
\usepackage[T1]{fontenc}
\usepackage{colortbl}
\usepackage{amsmath}
\usepackage{url}
\usepackage{xspace}
\usepackage[algo2e,titlenotnumbered,boxed,ruled,vlined,linesnumbered]{algorithm2e}
% the following package is optional:
%\usepackage{latexsym} 

% Following comment is from ijcai97-submit.tex:
% The preparation of these files was supported by Schlumberger Palo Alto
% Research, AT\&T Bell Laboratories, and Morgan Kaufmann Publishers.
% Shirley Jowell, of Morgan Kaufmann Publishers, and Peter F.
% Patel-Schneider, of AT\&T Bell Laboratories collaborated on their
% preparation.

% These instructions can be modified and used in other conferences as long
% as credit to the authors and supporting agencies is retained, this notice
% is not changed, and further modification or reuse is not restricted.
% Neither Shirley Jowell nor Peter F. Patel-Schneider can be listed as
% contacts for providing assistance without their prior permission.

% To use for other conferences, change references to files and the
% conference appropriate and use other authors, contacts, publishers, and
% organizations.
% Also change the deadline and address for returning papers and the length and
% page charge instructions.
% Put where the files are available in the appropriate places.

\title{Expert in the Crowd: Finding Influencers Using an Open-Ended Task}

% Single author syntax
\author{
    XI Lab
    \affiliations
    University of Fribourg, Switzerland\emails
    firstname.lastname@unifr.ch
}

% Multiple author syntax (remove the single-author syntax above and the \iffalse ... \fi here)
% Check the ijcai19-multiauthor.tex file for detailed instructions
\iffalse
\author{
First Author$^1$
\and
Second Author$^2$\and
Third Author$^{2,3}$\And
Fourth Author$^4$
\affiliations
$^1$First Affiliation\\
$^2$Second Affiliation\\
$^3$Third Affiliation\\
$^4$Fourth Affiliation
\emails
\{first, second\}@example.com,
third@other.example.com,
fourth@example.com
}
\fi
\newcommand{\iar}[1]{\textcolor{blue}{(*** @Ines: #1 ***)}}
\begin{document}

\maketitle

\begin{abstract}
\iar{write the abstract}
\end{abstract}

\section{Introduction}
Social influence is increasingly becoming a subtle force that controls the dynamics of the social network. 
Finding users in a network with a social influence is becoming a fundamental 
task for several applications such as brand marketing \cite{bond201261}
\cite{richardson2002mining} \cite{van2007new}, expert finding for question 
answering \cite{riahi2012finding} and collaborative task execution \cite{sun2014analyzing}\cite{miao2010generative}. These users are referred to as``influencers"
and are recognized with a set properties such as a high number of 
followers, a regular feed about a certain topic and an ``authentic" content.
 Existing work tackles the problem from a network perspective where 
 several approaches have been proposed to find influencers by performing content analysis and/or link analysis. However, the task of finding influencers remains a challenging task as these influencers
exhibit several subjective properties such as the ``authenticity" of their
content. Such a property is hard to assess by machines nowadays.
The current techniques used to find influencers in a social network
can be broadly classified into two groups:
\begin{itemize}
\item Influencer Maximization: Given a social network, the influence
maximization consists in finding the $k$ nodes who are 
most likely to trigger a lot of people adopting a new idea or product
by advertising it. Many efforts \cite{tang2015influence}  are currently focusing on designing and deploying scalable algorithms to find these $k$ nodes.
\item Expert identification: While the influencer maximization 
problem have been extensively studied, they are still not as reliable
as experts. For example, Zalando (a fashion online retailer)
relies on an influencer platform named ``Collabary"\footnote{\url{https://www.collabary.com}} to identify influencers.
\end{itemize}

This paper represents a step towards bridging the gap between these two classes by leveraging human workers for influencers identification. We consider a task where the crowd is asked
to name influencers, from where we collect a set of users 
recognized by the crowd as influencers. We introduce a new algorithm
called \textit{EinC} (Expert in the Crowd)
which combines social properties of the collected set of users  with 
worker's reliability to identify the real influencers in this set.

\subsection{Challenges and Contributions}


\subsection{Our Solution}



\section{Related Work}



\subsection{Aggregation Techniques }



\subsection{Social Influence Analysis}




\section{Method}

\subsection{Problem Statement}


\subsection{Model 1}

\subsection{Model 2}


\section{Experiments}

\subsection{Experimental Setup}

\subsection{Experiments about Model 1}

\subsection{Experiments about Model 2}

\section{Conclusion}



\section*{Acknowledgments}



%% The file named.bst is a bibliography style file for BibTeX 0.99c
\bibliographystyle{named}
\bibliography{ijcai19}

\end{document}

